\documentclass[12pt]{article}
\usepackage[utf8]{inputenc}
\usepackage{mathtools}
\usepackage{amsfonts}
\usepackage{stmaryrd}
\usepackage[colorlinks,linkcolor=black]{hyperref}

\title{Conversion des données CASD}
\author{Loïc Poullain, Augustin Wenger}
\date{\today}

\begin{document}

\maketitle

\begin{abstract}
Ce document présente les problématiques liées à l'utilisation des données fiscales,
à leur pré-traitement et à leur calcul. Il prend comme exemple le calcul
de l'impôt sur le revenu (IR) mais peut se généraliser à d'autres sujets.
\end{abstract}

\section{Introduction}

Grâce au \emph{centre d'accès sécurisé aux données (CASD)}, nous avons accès à l'ensemble des
déclarations fiscales des foyers fiscaux français. Ces données ne peuvent toutefois pas
être utilisées en direct dans nos produits.

La première contrainte est imposée par le respect du secret statistique. Les utilisateurs,
par leurs simulations, ne doivent pas être en mesure d'identifier une personne d'une
quelconque manière.

La deuxième est due à des objectifs de performance, les simulations devant
durer une minute toute au plus.
\newline

Pour pouvoir utiliser ces données et respecter ces contraintes,
nous devons les convertir en un autre ensemble, plus petit,
qui sera utilisé lors des simulations.

\section{Formulation formelle du problème}

Soit $E \subset \mathcal{F}(\mathbb{R}^n, \mathbb{R}^+)$ l'ensemble des modes de calcul (appelés aussi \emph{réformes})
possibles de l'impôt sur le revenu qui, aux données fiscales d'un foyer fiscal, associe
le montant de l'impôt de ce même foyer.
\newline

Quelques exemples de fonction $f$:
\begin{itemize}
  \item application d'un simple barème progressif ;
  \item application d'un barème proportionnel en fonction du nombre de parts ;
  \item etc.
\end{itemize}

Un exemple d'ensemble de départ $\mathbb{R}^n$:
\begin{itemize}
  \item revenu des activités salariés et situation maritale (veuf, célibataire, etc) ($n=2$).
\end{itemize}

Nous pouvons agir sur $n$ et la taille de $E$ (degré de liberté des réformes simulées).
\newline

Soit $\theta: E \longrightarrow \mathbb{R}^+$ une fonction qui,
à un mode de calcul de l'IR $f \in E$, associe
les recettes totales perçues par l'Etat pour cet impôt
\footnote{En pratique, $\theta$ est à valeurs dans un espace multidimensionnel.
Voir section \nameref{generalisation}.}.
\newline

En notant $\{ R_p \in \mathbb{R}^n / p \in  \llbracket 1, nombre de foyers fiscaux = P \rrbracket  \}$
l'ensemble des données fiscales de tous les foyers de France
\footnote{Dans les faits, le fichier exhaustif des déclarations contenant les feuilles d'impôt
d'environ 38 millions de foyers fiscaux, on utilise $P \approx 38M$.}
, il vient :

\begin{displaymath}
\theta = \theta_{P}: f \longmapsto \sum_{p=1}^{P} f(R_p)
\end{displaymath}

Lors du calcul de $\theta(f)$ pour $f \in E$, le simulateur n'a pas accès à $\{ R_p \}$ mais
à un ensemble plus petit $\{ R^*_j / j \leq J \leq P \}$ qui respecte le secret statistique et
que nous devons construire.
\newline

\textbf{On recherche donc un ensemble $\{ R^*_j \}$ et une fonction $\theta^{*}=\theta_J$
de la forme $\theta^{*}(f)=g(f, \{ R^*_j \})$ tel que $\forall f \in E, |\theta(f) - \theta^*(f)|$ soit minime.}
\newline

\textbf{La fonction $\theta^*$ doit être rapide à calculer (1 minute maximum).}

\section{Résolution numérique et complexité}

Dans le simulateur de \emph{Leximpact IR}, les restrictions actuelles sont les suivantes :

\begin{itemize}
  \item $E$ décrit l'ensemble des réformes \emph{paramétriques} de l'IR. Seules les valeurs
  numériques du texte de loi sont éditables.
  \item $n=1+n'$ et les données fiscales $\mathbb{R}^n=\mathbb{R}\times\prod_{i=1}^{n'} F_i$
  (où les $F_i$ sont des sous-ensembles de $\mathbb{R}$ ou $\mathbb{N}$) décrivent les révenus d'activité et la
  composition du foyer (situation maritale, nombre d'enfants à charge, caractéristiques des membres du foyer
  comme l'invalidité, etc).
\end{itemize}

On pose arbitrairement :

\begin{displaymath}
  \theta^*: f \longmapsto \sum_{p=1}^{J=?} f(R^*_j) \cdot w_j
\end{displaymath}

On recherche donc $J$ et $\{ R^*_j, w_j \}$ tels que $\forall f \in E, |\theta(f) - \theta^*(f)|$ soit minime.
\newline

Actuellement en pratique :
\begin{itemize}
  \item Le $J$ employé correspond au nombre de foyers fiscaux présents dans le fichier ERFS-FPR,
  qui nous permet d'utiliser ces données d'enquête comme base. Cette enquête contient les
  données de 50000 foyers fiscaux pondérés et représentatifs, ce qui représente une quantité
  permettant le calcul des effets d'une réforme dans openfisca en environ une minute,
  délai acceptable.
  \item Les fonctions $f$ utilisées ne sont pas des fonctions calculant l'impôt sur le revenu
  mais des fonctions dont l'agrégation sur la population française est disponible publiquement
  (ou dans les données exhaustives) : ``la part des foyers fiscaux ayant un revenu fiscal de
  référence supérieur à $x$'' pour une trentaine de valeurs de $x$.
  \item Après la calibration des revenus apparaissant dans l'enquête ERFS-FPR pour représenter
  la véritable distribution des RFR par foyer fiscal, une transformation affine par morceaux
  des résultats est effectuée pour prendre en compte des éléments qui n'apparaissent pas dans
  les données, principalement les crédits d'impôts. Cette transformation, qui s'effectue sur
  le serveur au moment du calcul, permet de faire correspondre les résultats obtenus par nos
  calculs avec le montant global d'impôts sur le revenu récolté par l'Etat.
\end{itemize}

\section{Généralisation du problème}
\label{generalisation}

Ce problème se généralise à d'autres sujets. Par exemple, pour le calcul de la CVAE
à l'échelle du pays, on a $\mathbb{R}^n=\mathbb{R}^2=\{ (CA, VA) \}$ où \emph{CA} est le chiffre d'affaires
et \emph{VA} la valeur ajoutée.

Plus généralement, en pratique, les fonctions $\theta$ et $\theta^*$ sont à valeurs dans $\mathbb{R}^m$.
Dans le cas de l'IR, par exemple, les valeurs de sortie sont le budget de l'Etat mais aussi
le nombre de personnes nouvellement imposables ou nouvellement exonérés, etc.

De la même manière, pour la CVAE, on pourrait, en sortie, classer les CVAE payés en fonction du secteur
d'activité, du type d'entreprises, etc.

\section{Pistes}

Plusieurs axes de recherche pour pouvoir supporter des $n$ plus grands tout en ayant un $\theta^*$ proche
de $\theta$ et qui se calcule en moins d'une minute:
\begin{itemize}
  \item augmenter $J$ et la puissance de calcul du serveur ;
  \item affiner la définition de $E$ pour réduire la taille de l'ensemble et jouer sur la structure de $f$ ;
  \item utiliser d'autres structures de $\theta^*$ ;
  \item trouver d'autres méthodes de calcul (efficaces notamment en complexité en temps) pour résoudre $\|\theta(f) - \theta^*(f)\|$ très petit.
\end{itemize}

\end{document}